\documentclass{acm_proc_article-sp}
\usepackage{url}
\usepackage{graphicx}
\begin{document}
\title{Sheetmusic: Making music from spreadsheets}
\numberofauthors{1}
\author{
\alignauthor
Thomas Levine\\
       \affaddr{csv soundsystem}\\
       \email{\_@thomaslevine.com}
}
\date{6 March 2014}
\maketitle
\begin{abstract}
The spreadsheet provides an intuitive paradigm for the expression of musical
scores. Musical scores can be expressed as data tables, with each record
corresponding to a place in time and each column corresponding to a note
or instrument. Sheetmusic is a plugin for Gnumeric that provides music
sequencing spreadsheet functions. Tools like Sheetmusic provide intuitive
music composition interfaces for people who are used to data analysis software.
Moreover, they help us plot data with the sense of sound.
\end{abstract}
\keywords{music, spreadsheets, gastronomification, data analysis}
\section{Introduction}
Data analyists often use visualization as a means for plotting data.
But there are other approaches!

\subsection{Music}
Ferguson \& al. \cite{ferguson} provide auditory analogs for simple
visual plots, such as the dotplot and boxplot.
% http://sonification.de/handbook/index.php/chapters/chapter8
Flowers \& Hauer (1992) \cite{flowers1992} created an auditory plot for
single-dimension datasets where the value of the variable of interest
is mapped to pitch and the resulting pitches are played in ascending
order. They had ten participants to study the same datasets with both
these auditory plots and conventional visual histograms and then to
judge statistical properties datasets. Accuracy was comparable between
the two plot types.

\subsection{Music videos}
Combining the visual and auditory senses, we can plot data in the form
of music videos. One example of this is the FMS Symphony
(csv soundsystem, 2012 \cite{fms}).
In the FMS Symphony, each beat of music corresponds to a business day
during the period between 2005 and 2013, the pitch of one instrument
corresponds to the United States interest rate, the pitch of another
instrument corresponds to the distance to the United States debt ceiling,
and the activation of certain flourishes corresponds to changes in the
balance of the United States treasury. These data are also represented
visually, through the combination of an animated line plot and a
Chernoff face.
\cite{fms-about}
% http://www.bdatafest.computationalreporting.com/projects/fms-symphony

Flowers \& al. \cite{flowers2005} propose that auditory plots may be
particularly helpful for augmenting visual information in instructional
settings, for enabling more complex inferences from data, and for
the analysis of spatial data.

\subsection{Food}
Why stop at just vision and hearing? We can plot data as food and use
all five senses.
Census Spices
% http://www.npr.org/blogs/codeswitch/2013/07/04/198424045/a-bbq-rub-that-tastes-like-brooklyn

\section{Our tools for musical plotting}
We at csv soundsystem \cite{csv} have been exploring multisensory
data plotting methods, including music videos and food.
In our production of data-driven music videos, we have recognized a need
for data analysis software and music software to be more strongly integrated.
We wanted a more seamless transition between modeling and music, and we
wanted it to be easier for data analysts to work with music. We have
developed tools like Sheetmusic to bridge this gap.

To use the language of the Grammar of Graphics, \cite{XXX}
we have abstract data and concrete plot elements,
and we define aesthetics that provide mappings
between the abstract data and the concrete elements.
The primitive plot elements that we use for music are things
like key, rhythm, pitch, and interval.

\subsection{Data tables}
We've found that the tabular representation of data aligns very well with
typical representations of music. Our data music tools work by mapping these
two concepts to each other.

We can think of data tables as collections of similar things, with the
same sorts of information being collected about each thing. In tidy data
tables \cite{tidydata} each row corresponds to an observation (a thing),
and each column corresponds to a variable. We add more rows to the table
as we observe more things, and we add more columns to the table as we
collect more information about each thing.

We can think of music as a composition of many different sounds over time,
with sounds coming from many different instruments. In musical scores
we represent time as movement from left to right, and we represent different
instruments by different staves (stacked on top of each other).
The staff becomes wider as the song gets longer, (They are often spread
across multiple pages.) and we add more staves as we add more instruments.

Rather than composing music as traditional sheet music,
we can use a table-editing program of our choice to compose
this sort of table. Our data music software simply adds
musical functions to table containers in various data analysis
tools. Sheetmusic is our offering for spreadsheets, but we
also have libraries for R data frames\cite{ddr} and Pandas
data frames.\cite{ddpy}










When we start using data analysis software for other things,
we also blur the line between data analysis and other things.
Data analysis seems very magical to many people. When we represent
data as familiar things like music, the analysis becomes more
accessible.

![abstract data](iris.png)

Numbers are abstract, theoretical things that don't have to be
related to the world. Given how strange numbers are, all of this
data stuff can seem very magical. In contrast, most people can
grasp the concept of guacamole quite well.

![concrete guacamole](data-guacamole.jpg)

Guacamole is a tasty but otherwise quite straightforward thing
that you can see and touch and taste and smell.

When I say that data things are "magical" to people, I mean that
people don't recognize that we can convert between these abstract
numbers and this real world. A big step in understanding this data
stuff is understanding how we make these conversions.

It turns out that the guacamole you see here is actually data-driven;
each bowl of guacamole represents New York City math test scores for
a different year, and the different ingredients correspond to the
average scores for students of different years in school. The bowl
to the right has more guacamole because of grade inflation---scores
went up across the board. One of the bowls (I don't remember which.)
is a bit spicier, because school years corresponding to pepper and
cilantro had relatively high grades that year.

<!--
p = ggplot(iris) + aes(x = Sepal.Width, y = Sepal.Length, color = Species) + geom_point(size = 5)                   
-->
![](iris-plot.png)

As I see it, food is just another way of plotting data. The people
in this room are probably more used to representing data with a plot like this.
This plot
might look very fancy, but we can decompose it into a bunch of simple
components.

* Each dot is an iris
* The distance from the bottom is how long the iris's petal is.
* The distance from the right is how wide the iris's petal is.
* The color of the dot is the species of iris.

To use the language of the Grammar of Graphics, we have abstract data
and concrete elements, and we define aesthetics that provide mappings
between the abstract data and the concrete elements.
And there's no reason that these elements have to be graphical!

    library(geomtaco)
    ggplot(iris) + aes(meat = Species, salsa = Sepal.Length) + geom_taco()

To convert
data into music or food, we just need to come up with parametrized primitive
musical or gastronomical elements. In this line of code, we use the geom\_taco
library to produce data-driven taco recipes with ggplot2.

When we produce data-driven music or food,
we are simply using a different plotting device to map abstract data
onto real things that we can perceive.









\section{How to use it}
Sheetmusic is implemented as a JavaScript function that loads data
from a Google Spreadsheet. Given the URL of a spreadsheet, Sheetmusic
downloads the spreadsheet contents and plays them in a browser.

The spreadsheet is organized as follows. Each column corresponds
to a musical track, and different tracks can have different music instruments.
Row corresponds to a beat (of time).
Each cell contains the frequency of sound to be played, represented
either as a number (Hertz) or in scientific notation (C4, D4, \&c.).

Sheetmusic accomplishes the playback of music, and this leaves
the data analyst to use conventional spreadsheet approaches for
composing music. For example, the following function can be used
to produce a major scale in a spreadsheet column.
\begin{verbatim}
# For example: ionian(440, 8)
function ionian(base, n) {
 var s = [0, 2, 4, 5, 7, 9, 11, 12]
 function freq(i) {
  return base*Math.pow(2,(Math.floor(i/12)+s[i])/12)
 }
 var scale=[]; for (var i=0; i<n; i++) scale.push(i)
 return scale.map(freq)
}
\end{verbatim}
Once you have a major scale in one column, you can easily make
chords with a spreadsheet function like this.
\begin{verbatim}
=A1*2^(4/12)
\end{verbatim}
If you put this in cell B1, A1 and B1 will form a major third interval
\cite{majorthird}.
You can read more about the determination of frequencies on Wikipedia \cite{piano}.

\section{Why make data-driven music}
We created sheetmusic and similar tools out of a need to produce
data-driven music, but I neglected to explain why we needed to do
that. Here are some of the uses that we have found for data-driven
music and music videos.

\subsection{Analyzing complex data}
Now that we're collecting so much data, we are reaching the limits of
data visualization. When produced and interpreted by capable people,
a good data visualization can represent about eight different variables.
If we want to visualize more variables than that, we must settle for
a reduced version of the data. By leveraging the sense of sound,
we expand our sensory bandwidth and enable the representation of
higher-dimensional data.

In the long term, we really need to gastronomify (turn into food)
data in order to experience them with all of the senses.
Unfortunately, that isn't feasible right now;
until we develop cheaper taste and smell APIs, we are stuck with what we have
on our smartphones, laptops, \&c., which is vision, hearing and touch. We need
to make data music videos in order to make the most of these tools.

\subsection{Reaching young people}
Combining data with music may also appeal to a younger audience.
According to the fictional eleven-year-old Emma Gertlowitz whose crush
recently switched from Justin Bieber to Nate Silver \cite{emma},
``[S]tatisticians are the new sexy vampires, only even more pasty.''
This example is just part of a larger trend: data is "in".

The White House used the appeal of data and music to advertise the State
of the Union Address; they published a video advertisement to YouTube that
used pie charts and dubstep, presumably to appeal to a younger audience \cite{whitehouse}.

\subsection{Education}
I find that a major hurdle in the understanding of quantitative disciplines
is an intuition of how to break complex concepts into discrete numbers.
I find that mapping numbers to things other than graphs gets people thinking
a bit more about what the numbers mean.

\subsection{People who can't see}
Data visualizations are typically not accompanied by an equivalent
alternative for people who can't see. We can redundantly express data
across multiple senses in order that people of varied ability can all
experience a particular data analysis.

\bibliographystyle{abbrv}
\bibliography{spreadsheets}
\balancecolumns
\end{document}
