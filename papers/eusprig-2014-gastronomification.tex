\documentclass{acm_proc_article-sp}
\usepackage{url}
\begin{document}
\title{Making music from spreadsheets}
\numberofauthors{1}
\author{
\alignauthor
Thomas Levine\\
       \affaddr{csv soundsystem}\\
       \email{\_@thomaslevine.com}
}
\date{6 March 2014}
\maketitle
\begin{abstract}
The spreadsheet is an intuitive paradigm for the expression of musical
scores. I'll discuss how you can turn data into music and show examples of
how csv soundsystem turns spreadsheets into data-driven music videos.
\end{abstract}
\section{Introduction}

\section{Relevance}


\subsection{Analyzing complex data}
Now that we're collecting so much data, we are reaching the limits of
data visualization. When produced and interpreted by capable people,
a good data visualization can represent about eight different variables.
If we want to visualize more variables than that, we must settle for
a reduced version of the data. By leveraging the sense of sound,
we expand our sensory bandwidth and enable the representation of
higher-dimensional data.

In the long term, we really need to gastronomify data in order to experience
them with all of the senses, but that isn't feasible right now.
Until we develop cheaper taste and smell APIs, we are stuck with what we have
on our smartphones, laptops, &c., which is vision, hearing and touch. We need
to make data music videos in order to make the most of these tools.

\subsection{Reaching young people}
According to Emma Gertlowitz\cite{emma},
``[S]tatisticians are the new sexy vampires, only even more pasty.''

Combining data with music may also appeal to a younger audience.
This is because both data and music are "in".

The White House used the appeal of data and music to advertise the State
of the Union Address; they published a video advertisement to YouTube that
used pie charts and dubstep to appeal to a younger audience.\cite{whitehouse}

\subsection{Eduction}
I find that a major hurdle in the understanding of quantitative disciplines
is an intition of how to break complex concepts into discrete numbers.
I find that mapping numbers to things other than graphs gets people thinking
a bit more about what the numbers mean.

\subsection{People who can't see}
Data visualizations are typically not accompanied by an equivalant
alternative for people who can't see. We can redundantly express data
across multiple senses in order that people of varied ability can all
experience a particular data analysis.
