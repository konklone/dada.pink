\documentclass{acm_proc_article-sp}
\begin{document}
\title{Automating the assessment of spreadsheet usability, data quality, and relevance}
\numberofauthors{1}
\author{
\alignauthor
Thomas Levine\\
       \email{\_@thomaslevine.com}
}
\date{6 March 2014}
\maketitle
\begin{abstract}
When you have lots of spreadsheets, it gets hard to look through all of them.
In my research, I have been exploring methods for understanding the contents
of thousands of spreadsheets at once. I will discuss strategies for automatically
assessing the usability of spreadsheets, the quality of data
in spreadsheets, and the relevance of specific spreadsheets to particular
analysis questions; I will explain both how these methods work and how they
can help you you manage and analyze your spreadsheets.
\end{abstract}

% A category with the (minimum) three required fields
%\category{H.4}{Information Systems Applications}{Miscellaneous}
%A category including the fourth, optional field follows...
%\category{D.2.8}{Software Engineering}{Metrics}[complexity measures, performance measures]
\terms{Research paper}
%\keywords{ACM proceedings, \LaTeX, text tagging} % NOT required for Proceedings

\section{Introduction}
Through recent open data initiatives, governments and large organizations have
begun releasing much of their internal data as public files on the internet.
From just a few different websites, we can easily download 100,000 different
spredsheets.%
\foot{Thomas Levine}{100,000 open data across 100 portal}{http://thomaslevine.com/!/data-about-open-data-talk-december-2-2013/}
My research began with this question: What becomes possible when we use
quantitative methods to look at 100,000 different datasets at once?

The research has since focused more specifically on trying to understand what
is going on in data-sharing ecosystems.
I've been collecting data about publically shared open data%
\foot{Thomas Levine}{Open Data had Better be Data-Driven}{http://thomaslevine.com/!/dataset-as-datapoint}
and looking for patterns in publishing and usage across the datasets.

\section{Related work}
Many other people have used relatively qualitative means to make sense of
the release of diverse open data spreadsheets.
For example, the Open Knowledge Foundation produced an
\foot{Open Knowledge Foundation}{Open Data Census}{http://census.okfn.org/}
of the availability of key datasets released by different countries.

McKinsey\foot
{James Manyika, Michael Chui, Diana Farrell, Steve Van Kuiken, Peter Groves, and Elizabeth Almasi Doshi, McKinsey Global Institute}
{Open data: Unlocking innovation and performance with liquid information}
{http://www.mckinsey.com/insights/business\_technology/open\_data\_unlocking\_innovation\_and\_performance\_with\_liquid\_information}
and the Governance Lab\foot
{Beth Simone Noveck}
{From Faith-Based to Evidence-Based: The Open Data 500 and Understanding How Open Data Helps the American Economy}
{http://www.forbes.com/sites/bethsimonenoveck/2014/01/08/from-faith-based-to-evidence-based-the-open-data-500-and-understanding-how-open-data-helps-the-american-economy/}%
\foot{Joel Gurin}{Open Data Now}{http://www.opendatanow.com/}
have looked at how specific businesses use specific publically available spreadsheets.

The general approach in these various studies is to look in depth at how a
few datasets are used, or how data-related projects are run. My research,
on the other hand, tries to get a broad picture across many different spreadsheets.



initiative. People have even codified all of this research into guidelines
about how to make data open.%
\foot{Sunlight Foundation}{Open Data Policy Guidelines}{http://sunlightfoundation.com/opendataguidelines/}%
\foot{Project Open Data}{Priciples}{http://project-open-data.github.io/principles/}%
\foot{Open Data Institute}{Open Data Certificate}{https://certificates.theodi.org}%
\foot{Open Knowledge Foundation}{Open Data Commons}{http://opendatacommons.org}%
\foot{Tim Berners-Lee}{Linked Data: Design Issues}{http://www.w3.org/DesignIssues/LinkedData.html}%
\foot{Open Government Working Group}{8 Principles of Open Government Data}{http://www.opengovdata.org/home/8principles}





\subsection{Citations}
Citations to articles \cite{bowman:reasoning, clark:pct, braams:babel, herlihy:methodology},
conference
proceedings \cite{clark:pct} or books \cite{salas:calculus, Lamport:LaTeX} listed
in the Bibliography section of your
article will occur throughout the text of your article.
You should use BibTeX to automatically produce this bibliography;
you simply need to insert one of several citation commands with
a key of the item cited in the proper location in
the \texttt{.tex} file \cite{Lamport:LaTeX}.
The key is a short reference you invent to uniquely
identify each work; in this sample document, the key is
the first author's surname and a
word from the title.  This identifying key is included
with each item in the \texttt{.bib} file for your article.

The details of the construction of the \texttt{.bib} file
are beyond the scope of this sample document, but more
information can be found in the \textit{Author's Guide},
and exhaustive details in the \textit{\LaTeX\ User's
Guide}\cite{Lamport:LaTeX}.

This article shows only the plainest form
of the citation command, using \texttt{{\char'134}cite}.
This is what is stipulated in the SIGS style specifications.
No other citation format is endorsed.

\subsection{Tables}
Because tables cannot be split across pages, the best
placement for them is typically the top of the page
nearest their initial cite.  To
ensure this proper ``floating'' placement of tables, use the
environment \textbf{table} to enclose the table's contents and
the table caption.  The contents of the table itself must go
in the \textbf{tabular} environment, to
be aligned properly in rows and columns, with the desired
horizontal and vertical rules.  Again, detailed instructions
on \textbf{tabular} material
is found in the \textit{\LaTeX\ User's Guide}.

Immediately following this sentence is the point at which
Table 1 is included in the input file; compare the
placement of the table here with the table in the printed
dvi output of this document.

\begin{table}
\centering
\caption{Frequency of Special Characters}
\begin{tabular}{|c|c|l|} \hline
Non-English or Math&Frequency&Comments\\ \hline
\O & 1 in 1,000& For Swedish names\\ \hline
$\pi$ & 1 in 5& Common in math\\ \hline
\$ & 4 in 5 & Used in business\\ \hline
$\Psi^2_1$ & 1 in 40,000& Unexplained usage\\
\hline\end{tabular}
\end{table}

To set a wider table, which takes up the whole width of
the page's live area, use the environment
\textbf{table*} to enclose the table's contents and
the table caption.  As with a single-column table, this wide
table will ``float" to a location deemed more desirable.
Immediately following this sentence is the point at which
Table 2 is included in the input file; again, it is
instructive to compare the placement of the
table here with the table in the printed dvi
output of this document.


\begin{table*}
\centering
\caption{Some Typical Commands}
\begin{tabular}{|c|c|l|} \hline
Command&A Number&Comments\\ \hline
\texttt{{\char'134}alignauthor} & 100& Author alignment\\ \hline
\texttt{{\char'134}numberofauthors}& 200& Author enumeration\\ \hline
\texttt{{\char'134}table}& 300 & For tables\\ \hline
\texttt{{\char'134}table*}& 400& For wider tables\\ \hline\end{tabular}
\end{table*}
% end the environment with {table*}, NOTE not {table}!

\subsection{Figures}
Like tables, figures cannot be split across pages; the
best placement for them
is typically the top or the bottom of the page nearest
their initial cite.  To ensure this proper ``floating'' placement
of figures, use the environment
\textbf{figure} to enclose the figure and its caption.

This sample document contains examples of \textbf{.eps}
and \textbf{.ps} files to be displayable with \LaTeX.  More
details on each of these is found in the \textit{Author's Guide}.

\begin{figure}
\centering
\epsfig{file=fly.eps}
\caption{A sample black and white graphic (.eps format).}
\end{figure}

\begin{figure}
\centering
\epsfig{file=fly.eps, height=1in, width=1in}
\caption{A sample black and white graphic (.eps format)
that has been resized with the \texttt{epsfig} command.}
\end{figure}


As was the case with tables, you may want a figure
that spans two columns.  To do this, and still to
ensure proper ``floating'' placement of tables, use the environment
\textbf{figure*} to enclose the figure and its caption.

Note that either {\textbf{.ps}} or {\textbf{.eps}} formats are
used; use
the \texttt{{\char'134}epsfig} or \texttt{{\char'134}psfig}
commands as appropriate for the different file types.

\subsection*{A {\secit Caveat} for the \TeX\ Expert}
Because you have just been given permission to
use the \texttt{{\char'134}newdef} command to create a
new form, you might think you can
use \TeX's \texttt{{\char'134}def} to create a
new command: \textit{Please refrain from doing this!}
Remember that your \LaTeX\ source code is primarily intended
to create camera-ready copy, but may be converted
to other forms -- e.g. HTML. If you inadvertently omit
some or all of the \texttt{{\char'134}def}s recompilation will
be, to say the least, problematic.

\section{Conclusions}

%
% The following two commands are all you need in the
% initial runs of your .tex file to
% produce the bibliography for the citations in your paper.
% \bibliographystyle{abbrv}
% \bibliography{sigproc}  % sigproc.bib is the name of the Bibliography in this case
% You must have a proper ".bib" file
%  and remember to run:
% latex bibtex latex latex
% to resolve all references
%
% ACM needs 'a single self-contained file'!
%
\balancecolumns
\end{document}
